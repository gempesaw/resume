% format two pieces of text, one left aligned and one right aligned
\newcommand{\headerrow}[3] {
\item[] \textbf{#1} \hfill #3}
%
\newenvironment{position}{
  \begin{itemize}[leftmargin=*]
    \setlength{\itemsep}{1pt}
    \setlength{\parskip}{0pt}
    \setlength{\parsep}{0pt}
}{\end{itemize}}

\section{Experience}

\subsubsection{Sharecare, Inc.}

\begin{position}
  \headerrow{QA Architect}{Sharecare, Inc.}{Aug 2012 - Present}
  \headerrow{Build Manager}{Sharecare, Inc.}{Oct 2011 - Aug 2012}
  \headerrow{QA Engineer I}{Sharecare, Inc.}{Aug 2011 - Oct 2011}
  %
  \paragraph{Honeydew}
  An in-house analog to Cucumber, Honeydew was a full-featured browser
  automation suite. It accepts feature files and executes
  cross-browser tests with Selenium. The backend was written in Perl
  and initially interfaced with browsers via Selenium RC. One of my
  first tasks was to rewrite the whole application to use Selenium
  Webdriver. I worked closely with Selenium::Remote::Driver, the Perl
  bindings for the Selenium Webdriver project, and even contributed a
  small fix to the CPAN module.

  \paragraph{} Honeydew includes a PHP/JavaScript front end that enabled any
  employee to write feature files in a DSL and test their
  functionality across all browsers. With jQuery I added autosuggests,
  instant feature searching, and per-user default settings to improve
  the user interface.
  \paragraph{Monitoring}
  I extended Honeydew with monitoring functionality that ran critical
  browser tests against our production websites to ensure 24/7
  functionality. The suite included email reports on failures and a UI
  for all members of the QA team to use, so everyone was able to
  assert critical functionality in their respective projects.
  \paragraph{Squash}
  I created a site crawler written in entirely in Perl that checks
  headers on URLs and all resources embedded within each page. Squash
  scanned three production sites and over 1.5 million pages each
  night, producing daily email reports with faulty static assets and
  storing load time and historical information in a database.
  \paragraph{REST API}
  In order to test REST APIs, I wrote a test suite using
  \href{https://github.com/SPORE}{SPORE} that would run validation
  tests on the REST APIs that our developers were creating. I also
  created a simple UI that enabled anyone to run the tests for each
  build.
  \paragraph{Manual QA Testing}
\end{position}
%%
%% \begin{myitem}
%% \item Created test framework for in-house REST API testing (Perl, PHP)
%% \item Created a customizable website-functionality monitoring suite (Perl, PHP)
%% \item Internal browser automation evangelist
%% \end{myitem}
%%
%% \begin{myitem}
%% \item Created a custom site crawler with daily email reports on resources for three websites (Perl)
%%   \begin{myitem}
%%   \item Crawled 1.5 million pages nightly, checking all referenced resources
%%   \item Aggregated email reports of status codes for urls, images, files
%%   \item Data is persisted in a MySQL DB and additional reporting available - status codes of a page over time, average load time for a resource
%%   \end{myitem}
%% \item Rewrote browser automation suite to use Webdriver 2.0 using Selenium::Remote::Driver,
%%   including minor contributions to Selenium::Remote::Driver (Perl)
%% \item Responsible for cross project interactions
%% \item Documentation: IE vm, regression documentation, critical path docs
%% \end{myitem}
%%
%% \begin{myitem}
%% \item Development of in-house browser automation tool using Selenium RC 1.0
%% \item Manual QA Testing of multiple websites across supported browsers/OS
%% \end{myitem}

%
\subsubsection{Georgia Institute of Technology}
%
\begin{position}
  \headerrow {Graduate Research Assistant}{Georgia Institute of Technology}{2008 - 2011}
  \begin{myitem}
  \item Co-authored \& refereed a conference paper: ``A Review of Wavelet-Based Algorithms for
    Applications in Reduced Order Modeling of Thermal Management Systems.''
  \item Presented at the NATO RTO-AVT-178 Specialists' Meeting on System Level Thermal Management
    for Enhanced Platform Efficiency: Bucharest, Romania.
  \item Conducted research on advanced strategies for data center cooling simulations
  \end{myitem}
\end{position}
